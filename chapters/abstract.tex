\chapter*{\centering Abstract}
\addcontentsline{toc}{chapter}{Abstract}

Electronic Control Units (ECU) in cars are the primary target in an attack on a car network. To assess the security of a particular network, a penetration tester must quickly and preferably automatically gather information about them. This can be accomplished by scanning ECUs with diagnostic protocols; most ECUs support at least one. The most widely supported diagnostic protocol is the Unified Diagnostic Services (UDS) protocol. The idea is to send every possible request and evaluate its response. This brute force approach leads to high run times. In this thesis, three approaches were evaluated to reduce the runtime by avoiding requests that are not expected to produce positive responses. They were elaborated based on collected data from eleven ECUs to find the best ways to implement these approaches and then realized for the UDS Scanner implemented in the Scapy library. Ultimately, they were found to provide high coverages and request savings resulting in new runtimes between 10\% — 30\% of the original runtimes. Nevertheless, complete coverage cannot be guaranteed.
