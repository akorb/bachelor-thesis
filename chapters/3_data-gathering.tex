\section{Data gathering and profiling}
\label{sec:data-gathering}

From here, the procedure and the environment of gathering information to elaborate approaches is described.

\subsection{Remote test environment}

In order to find scan ranges that are likely to be answered positively, data from as many ECUs as possible is needed. In the context of the PetS3 project a remote testing environment has been created. This was particularly helpful in light of the contact limitations at the time of writing.

The infrastructure of Laboratory for Safe and Secure Systems (LaS3) was used for that. Specifically, their GitLab server was used, displayed in \autoref{fig:gitlab-screenshot}.

%GitLab screenshot
\begin{figure}[h]
    \centering
    \includegraphics[width=0.7\textwidth]{gitlab-screenshot}
    \caption{Screenshot of the GitLab server.}
    \label{fig:gitlab-screenshot}
\end{figure}

Fourteen ECUs are available for testing on this server. However, three of them are from Opel and therefore only support GMLAN but not UDS.

For each ECU a YAML configuration file was created. Various work is done here, for example installing the CAN-ISOTP module, if the running device is not at least on Linux 5.10 (see \autoref{sec:scapy}). Also, the Scapy code from the currently working branch is pulled and checked out.

Then, the actual tests are executed that are written in the pytest framework. Here, a UDS scan can be started. At the same time, another process is started to record all CAN traffic that takes place during the execution of the tests.

Finally, the resulting files are uploaded to the GitLab server so that they can be downloaded via a browser program.

\subsection{Explaining the generated data of a scan}

In total, five files are stored after a scan, which will be explained in this section.

\begin{itemize}
    \item candump.log
    \item generic.log
    \item profiling.csv
    \item milestones.csv
    \item data.pkl
\end{itemize}

\subsubsection{Record communication between scanner and ECU}

The candump.log file is created by the candump program with the \mintinline{text}{-t} parameter. The generated files have a simple structure, each line representing one CAN packet.

\begin{samepage}
\begin{minted}{text}
(<timestamp>) <interface> <identifier>#<payload>
(1611779255.926425) can1 714#03225FB2CCCCCCCC
(1611779255.929936) can1 77E#037F2231AAAAAAAA
(1611779255.935338) can1 714#03225FB3CCCCCCCC
(1611779255.940680) can1 77E#037F2231AAAAAAAA
\end{minted}
\end{samepage}

Unfortunately, these files contain a lot of information that is not only not needed, but also unwanted, as it leads to more difficult analysis because of higher storage and processing power usage. \autoref{fig:can-unwanted-information} shows the information that is not needed because it is ECU specific and does not add any value for the analysis.

%CAN unwanted information
\begin{figure}[h]
    \centering
    \includegraphics[width=0.7\textwidth]{can-unwanted-information}
    \caption{ECU specific information in candump.log.}
    \label{fig:can-unwanted-information}
\end{figure}

Consequently, an abstraction is desirable. The newly created format is called the generic format. It removes the interface, the padding, resolves the transport protocol information (ISO-TP) and replaces the identifier with representatives (s = Server, c = Client). Each line represents a UDS packet, instead of a lower layer CAN packet as in the candump logs. \autoref{fig:candump-generic-conversion} illustrates the input and output of a conversion.

%candump to generic conversion
\begin{figure}[h]
    \centering
    \includegraphics[width=1\textwidth]{candump-generic-conversion}
    \caption{candump.log to generic.log conversion.}
    \label{fig:candump-generic-conversion}
\end{figure}

This format proved to be very useful as it allows quick analysis by chaining common Linux programs.
For instance, counting the number of requests of a scan is a single command:
\begin{samepage}
    \begin{minted}{bash}
        $ grep '^c.*' -c generic.log
        988844
    \end{minted}
\end{samepage}

\subsubsection{Store events and service scan runtimes of the scanner}

For profiling, two files have been added as output to the UDS scanner, namely profiling.csv and milestones.csv.

The former contains which enumerator ran under which state and when. It follows a fairly simple structure:

\begin{samepage}
    \begin{minted}{text}
        state     ,enumerator       ,start            ,end
        ECU Reset ,ECU Reset        ,1611770228.135198,1611770228.658480
        [session1],UDS_CCEnumerator ,1611770228.661016,1611770231.236505
        ECU Reset ,ECU Reset        ,1611770231.236789,1611770231.754865
        [session1],UDS_DSCEnumerator,1611770231.756320,1611770231.895722
    \end{minted}
\end{samepage}

Although an ECU Reset has no state, all occurrences of it get the same named state for better visualization of the log later.

Each line in the milestones.csv contains a timestamp and the name of the final state of a found state path and its timestamp. Hence, a state name can occur multiple times in this file, if there are more paths to the same state. What path is taken is decided later in the UDS scanner with finding the shortest path using the Dijkstra algorithm.

\begin{samepage}
    \begin{minted}{text}
        session1   ,1613302518.995168
        session2tp1,1613302569.605534
        session1tp1,1613309304.481464
        session3tp1,1613310988.231891
    \end{minted}
\end{samepage}

\subsubsection{Store the finished scanner object}

Last but not least, the data.pkl file. Pickle is the object serialization module of the Python Standard library \cite{pickle}. The data.pkl file is the serialized UDS Scanner object after the scan and thus contains all the results, consisting of the responses for each request for each state of an ECU. This is helpful for debugging, analysis and also for simulation.


\subsection{Profiling the UDS Scanner}

The Pareto principle indicates that small number of causes can be responsible for a large percentage of effects \cite{pareto}. This is also applicable to optimization problems. Usually only a few number of code lines cause the majority of the runtime. Optimizing them is far more effective than performing micro-optimizations as they naturally improve the performance to a greater extent. Moreover, micro-optimizations are usually even more difficult to accomplish.

\subsubsection{Visualizing UDS scans}

To profile the UDS Scanner, the profiling.csv and milestones.csv files are used. With their help Gantt charts are created which show the sequence and duration of enumerators within a scan. This visualizes what parts of the UDS Scanner cause most of its runtime.

These charts have been created for each available ECU. For illustration one is sufficient since each showed the same gist (see \autoref{fig:tesla-gantt}).

\begin{figure}[H]
    \centering
    \includegraphics[width=0.9\textwidth]{tesla-gantt}
    \caption{Gantt chart of the Tesla-Airbag-ECU. Each color is a state.}
    \label{fig:tesla-gantt}
\end{figure}


\subsubsection{Identifying the services causing the most runtime}

Although it seems that some enumerators and events do not require any time, their time consumption is only so small that they are not or hardly visible in this scaled-down version of the diagram.

The main cause of runtime is the RCEnumerator. This was expected, since this is the enumerator with the most generated requests.
In second place are the IOCBIEnumerator and the WDBISelectiveEnumerator.

The latter being a staged enumerator consisting of two enumerators, the RDBIEnumerator (Read data by identifier) and the WDBIEnumerator (Write data by identifier). The WDBI service expects the correct data length for each identifier. It is also necessary to ensure that the new data does not break the ECU, which is handled by writing the value of the RDBI service response, thus not changing the stored value. Hence, the RDBIEnumerator is the first stage to extract this information.

It also generates more requests than the WDBIEnumerator to a large extent, which accordingly hardly contributes to the runtime. The IOCBIEnumerator and RDBIEnumerator generate about the same runtime, this was to be expected since they generate the same number of requests. However, the IOCBI service is less frequently supported by ECUs than the RDBI service, resulting in a smaller runtime contribution when considering approach 3.

In summary, the RCEnumerator and the RDBIEnumerator are the most critical runtime causes of the UDS scanner and therefore belong to the matters to be optimized.
