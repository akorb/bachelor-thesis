\section{Conclusion and Future Work}

\subsection{Conclusion}

In this thesis, three approaches were evaluated and implemented to improve the efficiency of a UDS scan. Reusing information within the same service (approach 1), using probabilities of positive answers based on blocks (approach 2) and avoiding scanning unsupported services (approach 3). Simulations were performed to find the parameters that lead to the best results.

All three approaches turned out to maintain a high speed-up while still offering a high coverage.
Nevertheless, all three approaches are not able to provide a 100\% coverage at all times. 
If a coverage of 100\% is desired, a full scan might be the preferred solution, even if it takes much longer. This may be acceptable, if this scan will be executed only once. For fast results and good, but possibly not complete coverage, the new implementation should be preferred. It depends on the use case. Therefore, the new enumerators are not a replacement for the original ones, but an addition.

\subsection{Future Work}

Approach 2 could also be applied to the IOCBI service. Theoretically, to many more services, but most of them only have 8-bit identifiers, leading to almost no time savings, thus not worth lower coverages.

Also, another approach can be evaluated. While elaborating this work, it was observed that manufacturers tend to use the same identifiers across their ECUs for the RDBI service. So, these identifiers could be collected, grouped by manufacturers and used for more precise scanning through fixing some requests accordingly instead of only scanning randomly. One could go one step further here and not only group by manufacturers, but also by ECU type.
For example, all available BMW ECUs turned out to have 11 identifiers in common. On the other site, the both available BMW central gateway ECUs turned out to have 61 identifiers in common. But for this approach, more data is necessary. For this work were ten unique ECUs available, which support UDS, wherein only from BMW more than one ECU was available. This is the reason why this approach was not pursued further.
For example, it turned out that all available BMW ECUs have 11 identifiers in common. On the other hand, it turned out that the two available BMW Central Gateway ECUs have 61 identifiers in common. However, more data is needed for this approach. For this work, ten unique ECUs supporting UDS were available, with only BMW having more than one available. This is the reason why this approach was not pursued.

Approach 1 and approach 2 could be merged. For example, Type1 of the RC service is no longer scanned completely, but randomly based on probabilities. Unchanged is the second stage, where Type2 and Type3 are scanned based on the results of the first stage. This probably leads to even higher speed increases, but also to lower coverages.

In addition, the UDS Scanner and enumerators are not yet part of the official Scapy repository at press time. They will be refined and gradually added to the official repository.
